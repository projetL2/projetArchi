\documentclass{report}

\usepackage[latin1]{inputenc}
\usepackage[T1]{fontenc}
\usepackage[francais]{babel}
\usepackage{setspace}
\usepackage{listings} 
\usepackage{hyperref}
\usepackage{graphicx}


\begin{document}
\lstset{language=C}
\title{Compte-rendu final du projet \\Architecture des ordinateurs}
\author{Ga�tan CHAMBRES
\and
Chrystelle PETUREAU}
\date{10/05/2015}
\maketitle

\section*{Exercice 1:}
\subsection*{Question 1:}
On nous demande de faire en sorte que toutes les op�rations de type "iXX" soit confondu avec leur consoeurs. Pour cela, nous avons renommer l'identifiant de l'op�rateur I-ALUI en I-FREE1 et en pr�cisant qu'on d�fini maintenant I-ALUI comme la m�me chose que I-ALU avec la ligne(isa.h, ligne 29 et 33):
\begin{lstlisting}[frame=single]
#define I_ALUI I_ALU
\end{lstlisting}
De ce fait dans isa.c, on commente la partie (isa.c de la ligne 925 � la ligne 951)
\begin{lstlisting}[frame=single]
case I_ALUI:
\end{lstlisting}
pour ne pas avoir d'erreur lors de la compilation, vu que maintenant I-ALU et I-ALUI sont confondu.
\subsection*{Question 2:}
Pour permettre l'utilisation de I-OPL et OPL, voici les modifications apport�es dans les fichiers HCL:\\
Pour l'architecture seq, dans le ficher seq-std.hcl, aux lignes 129 et 130, nous avons ajout� ces deux lignes, qui permettent de v�rifier avant l'�tage EXECUTE si le registre srcA de l'op�ration OPL vaut rA ou RNONE. Si le registre vaut RNONE, alors on lit la valeur de la constante valC, sinon on lit valA, le contenu de rA.
\begin{lstlisting}[frame=single]
icode in {OPL} && rA == RNONE : valC;
icode in { OPL } : valA;
\end{lstlisting}

Nous avons alors test�,apr�s recompilation, avec le code suivant:
\begin{lstlisting}[frame=single]
	.pos 0
	irmovl 5,%eax
	irmovl 5,%ebx
	iaddl 5,%eax
	iaddl 5,%ebx
	addl %ebx,%eax
	halt
\end{lstlisting}
\subsection*{Question 3:}
\subsection*{Partie bonus:}

\section*{Exercice 2:}
On cherche � cr�er un pas � pas dans le fonctionnement du processeur.
\subsection*{Version s�quentielle:}
Dans le fichier seq-std.hcl, ligne 92, nous avons ajout� une instruction pour pr�ciser pour quelles valeurs le processeur passera � l'instruction suivante.
\begin{lstlisting}[frame=single]
int instr-next-ifun=[
	1:-1;
	];
\end{lstlisting} 
Dans le fichier ssim.c � la ligne 375, nous avons ajouter le prototype de la fonction "gen-instr-next-ifun" qui "lit" le compteur d'instructions. On ajoute cette fonction � la ligne 667.
\begin{lstlisting}[frame=single]
if(gen_instr_next_ifun () != -1)
	ifun = gen_instr_next_ifun();
else
\end{lstlisting} 

Enfin � la ligne 772, on ajoute l'instruction:
\begin{lstlisting}[frame=single]
if (gen_instr_next_ifun() == -1){
	pc_in = gen_new_pc();
}
\end{lstlisting} 
Elle permet de calculer la nouvelle valeur du compteur ordinal.
\subsection*{version pipe-lin�e:}
Les modifications sont quasiment similaire � la version s�quentielle.
Dans le fichier pipe-std.hcl, ligne 138, nous avons ajout� une instruction pour pr�ciser pour quelles valeurs le processeur passera � l'instruction suivante.
\begin{lstlisting}[frame=single]
int instr-next-ifun=[
	1:-1;
	];
\end{lstlisting} 
Dans le fichier psim.c � la ligne 1327, nous avons ajouter le prototype de la fonction "gen-instr-next-ifun" qui "lit" le compteur d'instructions. On ajoute cette fonction � la ligne 1361.
\begin{lstlisting}[frame=single]
if(gen_instr_next_ifun () != -1)
	if_id_next->ifun = gen_instr_next_ifun();
	fetch_ok= TRUE;
else
	fetch_ok=get_byte_val(mem,valp, &instr);
\end{lstlisting} 

Enfin � la ligne 1388, on ajoute l'instruction:
\begin{lstlisting}[frame=single]
if (gen_instr_next_ifun() == -1){
	pc_next->pc=gen_new_F_predPC();
}
\end{lstlisting} 
Elle permet de calculer la nouvelle valeur du compteur ordinal.

\section*{Exercice 3:}
\subsection*{Question 1:}
\subsection*{Question 2:}
\subsection*{Instruction mul:}
\subsection*{Instructions lods/stos/movs:}
\subsection*{Instruction repstos:}
\subsection*{Question 3:}
\subsection*{Question 4:}

\end{document}