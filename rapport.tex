\documentclass{report}

\usepackage[utf8]{inputenc}
\usepackage[T1]{fontenc}
\usepackage[francais]{babel}
\usepackage{setspace}
\usepackage{listings} 
\usepackage{hyperref}
\usepackage{graphicx}


\begin{document}
\lstset{language=C}
\title{Compte-rendu final du projet \\Architecture des ordinateurs}
\author{Guillaume ALMYRE 
\and
Ga�tan CHAMBRES
\and
Chrystelle PETUREAU}
\date{10/05/2015}
\maketitle

\section*{Exercice 1:}
\subsection*{Question 1:}
On nous demande de faire en sorte que toutes les op�rations de type "iXX" soit confondu avec leur consoeurs. Pour cela, nous avons renommer l'identifiant de l'op�rateur I-ALUI en I-FREE1 et en pr�cisant que I-ALU est la m�me chose que I-ALUI avec la ligne(isa.h, ligne 29 et 33):
\begin{lstlisting}[frame=single]
#define I_ALUI I_ALU
\end{lstlisting}
De ce fait dans isa.c, on commente la partie (isa.c de la ligne 925 � la ligne 951)
\begin{lstlisting}[frame=single]
case I_ALUI:
\end{lstlisting}
pour ne pas avoir d'erreur lors de la compilation, vu que maintenant I-ALU et I-ALU sont confondu.
\subsection*{Question 2:}
\subsection*{Question 1:}
\subsection*{Partie bonus:}

\section*{Exercice 2:}
\subsection*{Version squentielle:}
\subsection*{version pipe-lin�e:}

\section*{Exercice 3:}
\subsection*{Question 1:}
\subsection*{Question 2:}
\subsection*{Instruction mul:}
\subsection*{Instructions lods/stos/movs:}
\subsection*{Instruction repstos:}
\subsection*{Question 3:}
\subsection*{Question 4:}

\end{document}